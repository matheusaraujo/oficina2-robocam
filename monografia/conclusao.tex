\chapter{Conclusão}

Analisando a estrutura do robô desde a parte mecânica, elétrica e as estratégias para exploração e localização do objeto encontrado, esperamos que o robô consiga encontrar um objeto especifico e de destaque dentro de um ambiente pequeno, com as dimensões 168,2 cm por 118,9 cm.

Esperamos conseguir implementar corretamente um código de exploração inteligente baseado em representações locais para locomoção do robô dentro do espaço determinado. Além disso, esperamos conseguir aplicar um código para reconhecer o objeto a ser encontrado utilizando características de destaque (cores fortes, forma, etc.) do objeto. Tais características do robô (locomoção e identificação) serão controladas pela câmera acoplada ao robô, a CMUcam3.

Como continuação deste projeto, gostaríamos de ver o robô se locomovendo em espaços maiores e mais dinâmicos e também gostaríamos que o robô possa identificar objetos mais familiares ao dia a dia como algum equipamento ou molho de chaves. Se possível, também gostaríamos de ver o robô interagir com o ambiente a sua volta, como tentar buscar e pegar algum objeto ou simplesmente locomovê-lo. 
