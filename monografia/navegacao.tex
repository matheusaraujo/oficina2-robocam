\chapter{Navegação}

Para o robô encontrar o objeto anteriormente especificado ele deve saber se locomover dentro de um ambiente desconhecido. Para navegar no ambiente, o robô deve criar uma espécie de mapa que contenha informações que possibilitem sua locomoção no ambiente. Duas abordagens distintas são a representação de maneira global ou de maneira local \cite{conradt}.

De acordo com Jörg \cite{conradt}, a representação global procura descrever o ambiente da maneira mais fiel à realidade possível, indicando o mapa a partir de seus nodos e suas conexões, retendo o máximo de informação possível para conseguir manter uma consistência espacial global conforme o mapa vai sendo construído. A grande desvantagem da representação global é a necessidade de se reter e processar toda essa informação para criar-se o mapa, portanto conforme a quantidade de nodos aumenta, o mapa em si aumenta e a complexidade computacional também aumenta consideravelmente.

Na representação local os nodos (pontos de referência, denominados \textit{Place Agents}) possuem informações apenas do próprio local, de quais são os nodos vizinhos, possíveis pontos de interesse (objetos de destaque para identificação do nodo, por exemplo) e informações pertinentes sobre como alcançar o próximo nodo (direção, existência de obstáculos, etc.) seja este nodo já presente no mapa ou não.

Para o projeto atual, a equipe optou por utilizar a representação local, pois a abordagem global se torna ineficiente devido a sua complexidade computacional envolvida para manter a consistência do mapa. A abordagem local permite o escalonamento do sistema para permitir a ele operar em ambientes maiores.

\section{Construção do mapa}

Para a criação do mapa é utilizada principalmente a câmera \textit{CMUcam3} acoplada ao robô. A câmera é a principal responsável por adquirir informações do ambiente para a criação do mapa. Junto com a câmera, a bússola é utilizada para fornecer as direções entre um nodo e outro. Com as informações obtidas da câmera e da bússola, o robô é capaz de construir um mapa do ambiente e deve ser capaz de locomover-se dentro do ambiente, procurando o objeto desejado e desviando de obstáculos caso necessário.

Inicialmente, ao ser inicializado num novo ambiente, o robô deve criar um nodo A que será demarcado como o primeiro nodo existente no mapa. A primeira informação a ser obtida será procurar características do ambiente ou objetos próximos ao nodo para caracterizar o nodo em questão. Após obter informações do ambiente atual, o robô deve observar seus arredores e definir outros nodos alcançáveis a partir do nodo atual. O robô então deve guardar informações pertinentes sobre como chegar a cada um dos nodos vizinhos, como direção e a existência de objetos de destaque. Tais informações sobre o caminho serão guardadas em ambos os nodos envolvidos (nodo A e nodo B, inicio e destino, respectivamente). Dentre todos os nodos vizinhos, o sistema escolhe um e se locomove até ele. O processo então é repetido diversas vezes até que o mapa seja completamente montado (não existe nenhum nodo alcançável não visitado) ou o objeto seja encontrado.

\section{Roteamento}

Na representação local os nodos (pontos de referência, ou \textit{Place Agents}) possuem informações apenas do próprio local, de quais são os nodos vizinhos, possíveis pontos de interesse (objetos de destaque para identificação do nodo, por exemplo) e informações pertinentes sobre como alcançar o próximo nodo (direção, existência de obstáculos, etc.) seja este já presente no mapa ou não. Na representação local, para se alcançar um nodo já mapeado e que não seja vizinho ao atual será necessário um algoritmo de roteamento no sistema, onde cada \textit{Place Agent} pergunta por essa informação aos seus vizinhos e assim sucessivamente até que o nodo requisitado seja encontrado.

Quando o nodo requisitado é encontrado o robô traça um caminho para chegar ao local desejado, passando por nodos já existentes. 
