%% Exemplo de utilizacao do estilo de formatacao normas-utf-tex (http://normas-utf-tex.sourceforge.net)
%% Autores: Hugo Vieira Neto (hvieir@utfpr.edu.br)
%%          Diogo Rosa Kuiaski (diogo.kuiaski@gmail.com)
%% Colaboradores:
%%          Cézar M. Vargas Benitez <cesarvargasb@gmail.com>
%%          Marcos Talau <talau@users.sourceforge.net>


%\documentclass[openright]{normas-utf-tex} %openright = o capitulo comeca sempre em paginas impares
\documentclass[oneside]{normas-utf-tex} %oneside = para dissertacoes com numero de paginas menor que 100 (apenas frente da folha) 

\usepackage[alf,abnt-emphasize=bf,bibjustif,recuo=0cm, abnt-etal-cite=2, abnt-etal-list=99]{abntcite} %configuracao correta das referencias bibliograficas.

\usepackage[brazil]{babel} % pacote portugues brasileiro
\usepackage[utf8]{inputenc} % pacote para acentuacao direta
\usepackage{amsmath,amsfonts,amssymb} % pacote matematico
\usepackage{graphicx} % pacote grafico
\usepackage{times} % fonte times

%Podem utilizar GEOMETRY{...} para realizar pequenos ajustes das margens. Onde, left=esquerda, right=direita, top=superior, bottom=inferior. P.ex.:
%\geometry{left=3.0cm,right=1.5cm,top=4cm,bottom=1cm} 

% ---------- Preambulo ----------
\instituicao{Universidade Tecnol\'ogica Federal do Paran\'a} % nome da instituicao
\programa{Departamento Acadêmico de Eletrônica} % nome do programa
\area{Inform\'atica Industrial} % [Engenharia Biom\'edica] ou [Inform\'atica Industrial] ou [Telem\'atica]

\documento{Monografia} % [Disserta\c{c}\~ao] ou [Tese]
\nivel{Mestrado} % [Mestrado] ou [Doutorado]
\titulacao{Mestre} % [Mestre] ou [Doutor]

\titulo{\MakeUppercase{Robô Explorador de Ambientes}} % titulo do trabalho em portugues
\title{\MakeUppercase{Ambience Explorer Robot}} % titulo do trabalho em ingles

\autor{Luis Guilherme Machado Camargo} % autor do trabalho
\autordois{Marcelo Teider Lopes}
\autortres{Matheus Silva Araújo}
\cita{CAMARGO, Luis Guilherme M. ; LOPES, Marcelo Teider; ARAÚJO, Matheus Silva} % sobrenome (maiusculas), nome do autor do trabalho

\palavraschave{Robótica, Exploração, Reconhecimento de Imagens, Sensores ...} % palavras-chave do trabalho
\keywords{Robotics, Exploration, Image Recognition, Sensors ...} % palavras-chave do trabalho em ingles

%\comentario{\UTFPRdocumentodata\ apresentada ao \UTFPRdocumentodata\ da \ABNTinstituicaodata\ como requisito parcial para obten\c{c}\~ao do grau de ``\UTFPRtitulacaodata\ em Ci\^encias'' -- \'Area de Concentra\c{c}\~ao: \UTFPRareadata.}

\comentario{\UTFPRdocumentodata\ apresentada ao Departamento Acadêmico de Eletrônica da \ABNTinstituicaodata\ como requisito parcial para aprovação na Disciplina de Oficina de Integração 2.}


\orientador[Orientadora:]{Profa. Dra. Myriam Regattieri De Biase da Silva Delgado} % nome do orientador do trabalho
%\orientador[Orientadora:]{Nome da Orientadora} % <- no caso de orientadora, usar esta sintaxe
%\coorientador{Nome do Co-orientador} % nome do co-orientador do trabalho, caso exista
%\coorientador[Co-orientadora:]{Nome da Co-orientadora} % <- no caso de co-orientadora, usar esta sintaxe
%\coorientador[Co-orientadores:]{Nome do Co-orientador} % no caso de 2 co-orientadores, usar esta sintaxe
%\coorientadorb{Nome do Co-orientador 2}	% este comando inclui o nome do 2o co-orientador

\local{Curitiba} % cidade
\data{\the\year} % ano automatico


%---------- Inicio do Documento ----------
\begin{document}

\capa % geracao automatica da capa
\folhaderosto % geracao automatica da folha de rosto
%\termodeaprovacao % <- ainda a ser implementado corretamente

% dedicatória (opcional)
%\begin{dedicatoria}
%Texto da dedicat\'oria.
%\end{dedicatoria}

% agradecimentos (opcional)
\begin{agradecimentos}
Este trabalhado não teria sido possível sem o projeto anteriormente apresentado por Bruno Meneguele, Fernando Padilha e Vinicius Arcanjo.
Por emprestar o robô e pelos diversos esclarecimentos (muitas vezes sobre assuntos que não os envolviam) nosso muito obrigado.

À Professora Myriam nosso agradecimento por aceitar o desafio de nos orientar e a atenção dispensada. 

Aos Professores Hugo Vieira e Mário Sérgio pela oportunidade sem par de aprendizado.

Aos Professores João Fabro e Celso Kaestner por permitirem o uso das dependências e recursos do Laboratório de Arquitetura de Computadores para a realização do projeto.

Aos colegas Lucas Campiolo Paiva e Cláudio Akio pela colaboração ocasional com o projeto.

Aos marceneiros do Almoxarifado da UTFPR pela ajuda com a construção do suporte da câmera.
 
\end{agradecimentos}

% epigrafe (opcional)
\begin{epigrafe}
\begin{itemize}
	\item \textbf{1ª lei:} Um robô não pode ferir um ser humano ou, por inacção, permitir que um ser humano sofra algum mal.
	\item \textbf{2ª lei:} Um robô deve obedecer às ordens que lhe sejam dadas por seres humanos, exceto nos casos em que tais ordens contrariem a Primeira Lei.
	\item \textbf{3ª lei:} Um robô deve proteger sua própria existência, desde que tal proteção não entre em conflito com a Primeira e Segunda Leis.
	\newline
	\item \textbf{Lei Zero:} Um robô não pode fazer mal à humanidade e nem, por inacção, permitir que ela sofra algum mal.
	\newline
	\newline
	\textit{Isaac Asimov}
\end{itemize}

\end{epigrafe}

%resumo
\begin{resumo}
Neste projeto foi realizada uma tentativa de desenvolver um robô explorador ambientes capaz de encontrar um objeto pré-definido em um ambiente controlado, ou explorar todo o ambiente caso não seja capaz de encontrá-lo. Posteriormente, o escopo do projeto foi alterado para um robô capaz de localizar um objeto em seu campo de visão e seguí-lo, como passo inicial para a construção de um robô explorador. Para tal usamos como sensor principal uma câmera, a \textit{CMUCam3}, desenvolvida pela \textit{Carmegie Mellon University}. Uma bússola também foi estudada como sensor auxiliar, para obter a direção do robô. A versão inicial do robô foi desenvolvido em \cite{Robo2d}.
\end{resumo}

%abstract
\begin{abstract}
An attempt to develop an explorer robot is made in this project, which would be capable of both finding a predefined object on a small, controlled, environment, and exploring the whole environment if it can't find it. Later the scope of the project was changed to a robot capable of finding an object at his line of vision and following it, as a inicial step in order to build an explorer robot.  A camera was used as the main sensor, the \textit{CMUCam3}, developed by \textit{Carmegie Mellon University}. A compass would also be used as an auxiliary sensor, in order to obtain the direction of the robot. The initial version of the Robot is developed in \cite{Robo2d}.
\end{abstract}

% listas (opcionais, mas recomenda-se a partir de 5 elementos)
\listadefiguras % geracao automatica da lista de figuras
\listadetabelas % geracao automatica da lista de tabelas
%\listadesiglas % geracao automatica da lista de siglas
%\listadesimbolos % geracao automatica da lista de simbolos

% sumario
\sumario % geracao automatica do sumario


%---------- Inicio do Texto ----------
% recomenda-se a escrita de cada capitulo em um arquivo texto separado (exemplo: intro.tex, fund.tex, exper.tex, concl.tex, etc.) e a posterior inclusao dos mesmos no mestre do documento utilizando o comando \input{}, da seguinte forma:
%\input{intro.tex}
%\input{fund.tex}
%\input{exper.tex}
%\input{concl.tex}


% ========== %
% INTRODUÇÃO %
% ========== %

\chapter{Introdução}

\section{Motivação}


\section{Objetivo}


\subsection{Objetivo Geral}


\subsection{Objetivos Específicos}


\section{Visão Geral do Projeto}


% ================ %
% DESCRIÇÃO DO ROBÔ %
% ================ %

\chapter{Sistema Mecânico}

\section{Projeto Mecânico}

\section{Plataforma Arduíno}


% ======== %
% SENSORES %
% ======== %

\chapter{Sensores}
\label{sec_sensores}

A equipe utiliza dois sensores no projeto: um eletromagnético digital, Bússola; e um ótico, Câmera. O primeiro fornece informação de orientação do robô e o segundo informações visuais do ambiente.

\section{Bússola}
\label{sec_bussola}

A bússola utilizada no projeto é a \textit{Dinsmore Sensor Modelo \# 1490} \cite{bussola}, emprestada à equipe pelo professor Hugo Vieira.

Foi necessário construir um \textit{shield} para conectá-la ao \textit{Arduíno}. O circuito utilizado é apresentado na figura \ref{sen_fig01} e foi baseado no trabalho do \textit{Autonomous Vehicle Team} do \textit{College of New Jersey} \cite{newjersey}.

Internamente a bússola funciona como um transistor coletor aberto NPN e fornece sinais 0 ou 5V, TTL, \cite{bussola}, captados nos catodos dos diodos, para as entradas digitais do \textit{Arduíno}, sendo alimentada com 5V através do \textit{Arduíno} também.

\begin{figure}[h!]
    \center
    \includegraphics[scale=0.4]{imagens/circuito_bussola.png}
    \caption{Circuito Eletrônico - Bússola}
    \label{sen_fig01}
\end{figure}

A combinação dos quatro sinais fornecidos pela bússola fornece sua orientação com uma precisão de 45 graus, \textit{i.e.}, há oito estados possíveis para a bússola. 

A combinação entre esses estados e sua representação são apresentados na Tabela \ref{sen_tbl01}. As combinações de \textit{bits} que não representam um estado válido não são apresentadas.

\begin{table}[h!]
    \centering
    \begin{tabular}{|c|c|c|c|c|c|} \hline
        \textbf{Sinal 1} & \textbf{Sinal 2} & \textbf{Sinal 3} & \textbf{Sinal 4} & \multicolumn{2}{|c|}{\textit{Direção}} \\ \hline
        0 & 0 & 0 & 1 & W & Oeste \\ \hline
        0 & 0 & 1 & 0 & S & Sul \\ \hline
        0 & 0 & 1 & 1 & SW & Sudoeste \\ \hline
        0 & 1 & 0 & 0 & E & Leste \\ \hline
        0 & 1 & 1 & 0 & SE & Sudeste \\ \hline
        1 & 0 & 0 & 0 & N & Norte \\ \hline
        1 & 0 & 0 & 1 & NW & Noroeste \\ \hline
        1 & 1 & 0 & 0 & NE & Nordeste \\ \hline
    \end{tabular}
    \caption{Acionamento dos Motores}
    \label{sen_tbl01}
\end{table}

\section{Câmera}
\label{sec_camera}

Para o projeto foi adquirida uma \textit{CMUcam3}, desenvolvida pela \textit{Carmegie Mellon University}, que se propõe a criar um sistema de visão simples em sistemas embarcados através de um sensor inteligente \cite{cmucam01}.

A \textit{CMUcam3} utiliza um sistema baseado na arquitetura \textit{ARM7TDMI} e tem como principal microprocessador um \textit{Philips LPC2106} conectado a uma câmera CMOS \cite{cmucam02}.

A justificativa da escolha da \textit{CMUcam3} se baseia na existência de diversos exemplos de códigos e bibliotecas prontas para o processamento de imagens para essa plataforma, como obtenção de mapa de cores, histograma e detecção de bordas.

Por se tratar de um microprocessador com maior capacidade de processamento, o sistema de controle e tomada de decisões foi desenvolvido no \textit{LPC2106}.

A alimentação da \textit{CMUcam3} é feita por quatro pilhas AA em série, totalizando 6V. 




% ===== %
% VISÃO %
% ===== %

\chapter{Visão}

\section{Reconhecimento de Imagens}


% ========= %
% NAVEGAÇÃO %
% ========= %

\chapter{Navegação}

\section{Place Agents}

\section{Construção de Mapa}

\section{Roteamento}


% ========= %
% CONCLUSÃO %
% ========= %

\chapter{Conclusão}

Analisando a estrutura do robô desde a parte mecânica, elétrica e as estratégias para exploração e localização do objeto encontrado, esperamos que o robô consiga encontrar um objeto especifico e de destaque dentro de um ambiente pequeno, com as dimensões 168,2 cm por 118,9 cm.

Esperamos conseguir implementar corretamente um código de exploração inteligente baseado em representações locais para locomoção do robô dentro do espaço determinado. Além disso, esperamos conseguir aplicar um código para reconhecer o objeto a ser encontrado utilizando características de destaque (cores fortes, forma, etc.) do objeto. Tais características do robô (locomoção e identificação) serão controladas pela câmera acoplada ao robô, a CMUcam3.

Como continuação deste projeto, gostaríamos de ver o robô se locomovendo em espaços maiores e mais dinâmicos e também gostaríamos que o robô possa identificar objetos mais familiares ao dia a dia como algum equipamento ou molho de chaves. Se possível, também gostaríamos de ver o robô interagir com o ambiente a sua volta, como tentar buscar e pegar algum objeto ou simplesmente locomovê-lo. 


%---------- Referencias ----------
\bibliography{monografia} % geracao automatica das referencias a partir do arquivo reflatex.bib

%---------- Apendices (opcionais) ----------
\apendice

\chapter{Caderno de Bordo}

\textbf{10 de Agosto}

Primeira aula da disciplina. 

\textbf{17 de Agosto}

Marcelo Teider e Matheus Araujo decidem formar a equipe, têm em mente o projeto de um robô explorador. A equipe ainda não tem os três integrantes, como sugerido para a Disciplina. Então iniciam-se negociações com outras equipes para a definição dos integrantes.

\textbf{12 de Agosto}


Após algumas conversas, as equipes da disciplina são definidas. Luis Camargo é integrado à equipe. 

\textbf{13 de Agosto}

Após a definição da equipe, a professora Myriam Delgado é convidada para nos orientar, aceitando a proposta. 

\textbf{13 de Agosto}

O projeto é então definido como a construção de um robô explorador. A intenção é “mostrar” ao robô um objeto para reconhecimento, então esse mesmo objeto é escondido de seu campo de visão e o robô deve então explorar o ambiente procurando-o. Ele deve também evitar obstáculos durante o percurso.

\textbf{24 de Agosto}

Apresentação da pré-proposta. A equipe apresentou aos professores da disciplina a pré-proposta de projeto, sendo aceita pelos professores. Como sugerido pelo professor Hugo durante a apresentação, decidimos comprar uma \textit{SmartCam}. A primeira ideia seria utilizar a \textit{CMUcam}, mas optamos por pesquisar outros modelos. 

\textbf{29/30 de Agosto}

Iniciamos as pesquisas das câmeras. Pesquisamos os modelos \textit{AVRcam}, que não está sendo produzido no momento; e alguns modelos comerciais, que foram descartados devido ao elevado custo, acima de US\$ 2000,00. Dentre os modelos \textit{CMUcam}, ficamos em dúvida entre dois modelos, \textit{CMUcam1 }e \textit{CMUcam2}. 

Fizemos um levantamento do material que será necessário para a construção do robô, como motor, chassi, bateria e microprocessador. 

\textbf{31 de Agosto}

Em conversa com o professor Hugo, optamos pela \textit{CMUcam1}; uma vez que ela atende os requisitos do projeto e tem menor custo. Os procedimentos para adquirá-la foram tomados.

Conseguimos com a equipe que desenvolveu o Robô Explorador de Labirintos 2D nessa mesma disciplina, em 2011-01, o robô emprestado. Segundo a Equipe, uma das Pontes H do robô apresenta problemas, precisaremos solucioná-lo. 

Precisamos agora decidir e conseguir o microcontrolador para o Robô. Os da plataforma \textit{Arduíno} apresentam possibilidade de comunicação com a câmera via software e pela familiaridade dos membros da equipe podem se tornar uma boa escolha.

\textbf{5 de Setembro}

Pegamos o robô com a equipe do semestre passado. Sem problemas com a ponte H, o robô não apresenta nenhum problema mecânico ou elétrico-eletrônico. Estamos utilizando o mesmo microcontrolador da equipe, um \textit{ATMEGA 328P} em  uma placa \textit{Arduíno Duemilanove}. 

\textbf{ 7 de Setembro}

Desenvolvemos uma API para controle do hardware do Robô. Fizemos uma classe \textit{Robot} em C++ com funções pré-definidas como \textit{startrobot}, \textit{forward}, \textit{turnleft}. No código a ser desenvolvido não necessitaremos controlar o robô diretamente, apenas através dessa classe.

\textbf{12 de Setembro}

Por indicação de nossa orientadora, conhecemos o trabalho \textit{A Distributed Cognitive Map for Spatial Navigation Based on Graphically Organized Place Agents}, de Jörg Conradt e Rodney Douglas, e por orientação do professor Luiz Merkle com um algoritmo de busca recursiva em um espaço bidimensional pelas quatro direções cardeais através de pilha, no livro \textit{Data Structures - An advanced approach using C}.

\textbf{21 de Setembro}

Definida a data da banca final, 7 de Dezembro. 

Por orientação do professor Hugo, entramos em contato com diversos artigos de Prestes, Lowe, Bay e Artolazabal.

\textbf{23 a 28 de Setembro}

Escrita do relatório de Qualificação, entregue aos professores no dia 28. Percebemos a necessidade de usar uma bússola como sensor para definir a direção em que se encontra o robô.

\textbf{28 de Setembro}

Entrega da Qualificação aos professores da Disciplina. Conseguimos emprestado com o professor Hugo duas bússolas \textit{Dinsmore}, uma digital e outra analógica. Durante a próxima semana faremos os testes para definir qual iremos usar. A bússola será integrada ao sistema através da placa do \textit{Arduíno}.

\textbf{30 de Setembro}

Foram encontrados problemas com o interfaceamento entre a bússola e o arduíno devido a diferentes níveis de tensão e corrente, será necessário construir um conversor DC-DC para fazer a integração.

\textbf{6 de Outubro}

Após orientação do professor Hugo e pesquisa de outros projetos, fizemos o primeiro experimento com a bússola e obtivemos 5 [V] na saída, o que possibilita interfaceamento direto com o arduíno. 

\textbf{12 de Outubro}

Construímos um \textit{shield} para a bússola, possibilitando conectá-la ao Arduíno.

\textbf{14 de Outubro}

A câmera foi finalmente enviada de Hong Kong. A previsão de entrega é de um mês, precisamos elaborar um plano B caso a câmera não chegue a tempo da finalização do projeto.

\textbf{15 de Outubro}

Conectamos o \textit{shield} da bússola ao Arduíno; tivemos alguns problemas de mau contato com os cabos, mas o robô é capaz de interpretar as informações fornecidas pela bússola e segue os comandos que lhe foram dados. 

\textbf{18 de Outubro}

Tivemos uma reunião com nossa orientadora, Prof. Myriam, onde apresentamos o estado atual do projeto e elaboramos um segundo plano, caso a câmera não chegue. Iremos utilizar os sensores do projeto passado, cinco pares de leds emissor/receptor, e construir uma pasta com marcas de diferentes cores perceptíveis pelo robô, onde ele deverá construir um mapa cognitivo. 

\textbf{19 de Outubro}

A câmera chegou! Iremos começar os testes com ela em breve. Precisaremos de um cabo serial ou um adaptador para comunicação com a placa e também de baterias para alimentação da mesa, 6V.

\textbf{26 de Outubro}

Compramos um adaptador para o cabo de comunicação e um suporte para quatro pilhas AA que servirá como alimentação.

Conseguimos compilar e gravar os códigos de exemplo na câmera.

\textbf{2 de Novembro}

Estamos estudando a comunicação da \textit{CMUCam} com o \textit{Arduíno}.

\textbf{9 de Novembro}

Concentramos nossas atenções na finalização da primeira versão da monografia.


\include{descritivo}

%\chapter{Nome do Ap\^endice}

%Use o comando {\ttfamily \textbackslash apendice} e depois comandos {\ttfamily \textbackslash chapter\{\}}
%para gerar t\'itulos de ap\^en-dices.


% ---------- Anexos (opcionais) ----------
%\anexo
%\chapter{Nome do Anexo}

%Use o comando {\ttfamily \textbackslash anexo} e depois comandos {\ttfamily \textbackslash chapter\{\}}
%para gerar t\'itulos de anexos.


% --------- Lista de siglas --------
%\textbf{* Observa\c{c}\~oes:} a lista de siglas nao realiza a ordenacao das siglas em ordem alfabetica
% Em breve isso sera implementado, enquanto isso:
%\textbf{Sugest\~ao:} crie outro arquivo .tex para siglas e utilize o comando \sigla{sigla}{descri\c{c}\~ao}.
%Para incluir este arquivo no final do arquivo, utilize o comando \input{arquivo.tex}.
%Assim, Todas as siglas serao geradas na ultima pagina. Entao, devera excluir a ultima pagina da versao final do arquivo
% PDF do seu documento.


%-------- Citacoes ---------
% - Utilize o comando \citeonline{...} para citacoes com o seguinte formato: Autor et al. (2011).
% Este tipo de formato eh utilizado no comeco do paragrafo. P.ex.: \citeonline{autor2011}

% - Utilize o comando \cite{...} para citacoeses no meio ou final do paragrafo. P.ex.: \cite{autor2011}



%-------- Titulos com nomes cientificos (titulo, capitulos e secoes) ----------
% Regra para escrita de nomes cientificos:
% Os nomes devem ser escritos em italico, 
%a primeira letra do primeiro nome deve ser em maiusculo e o restante em minusculo (inclusive a primeira letra do segundo nome).
% VEJA os exemplos abaixo.
% 
% 1) voce nao quer que a secao fique com uppercase (caixa alta) automaticamente:
%\section[nouppercase]{\MakeUppercase{Estudo dos efeitos da radiacao ultravioleta C e TFD em celulas de} {\textit{Saccharomyces boulardii}}
%
% 2) por padrao os cases (maiusculas/minuscula) sao ajustados automaticamente, voce nao precisa usar makeuppercase e afins.
% \section{Introducao} % a introducao sera posta no texto como INTRODUCAO, automaticamente, como a norma indica.


\end{document}
