\chapter{Introdução}

Robôs autônomos são aqueles capazes de realizar um objetivo desejado em um determinado ambiente sem a intervenção humana. Dentre os robôs autônomos, robôs exploradores são aqueles que têm como objetivo explorar um ambiente até atingir uma finalidade específica, um ponto ou um objeto dentro do espaço a ser explorado, por exemplo, sem se perder ou colidir com obstáculos.

Para atingir seus objetivos, esses robôs utilizam sensores para percepção do ambiente e algoritmos inteligentes para tomada de decisões.

\section{Motivação}

Robôs autônomos exploradores têm diversas áreas de aplicação, desde atividades em hospitais a explorações planetárias. 

A possibilidade de aplicar diversas áreas de conhecimento do curso, de eletrônica à inteligência artificial, motivou a equipe a escolher esse projeto. O subsídio de um robô já desenvolvido e a possibilidade de utilizar um sensor inteligente (\textit{CMUcam3}) viabilizaram o projeto.

\section{Objetivo}

O objetivo do projeto é a construção de um robô explorador, utilizando uma câmera e uma bússola como sensores. Ele deve ser capaz de identificar um objeto específico e posteriormente explorar o ambiente onde se encontra à procura do objeto, caso este saia de seu campo de visão.

\subsection{Objetivo Geral}

Para o projeto atual, o ambiente de exploração será limitado a um retângulo de 168,2 cm por 118,9 cm, dimensões somadas de duas folhas tamanho A0, que serão utilizadas lado a lado para composição da arena de exploração. A cor branca das folhas irá determinar o chão para o robô. Eventualmente, podem ser colocados referenciais de fácil identificação (círculos de determinadas cores, por exemplo) nas extremidades da arena para melhor auto-localização do robô.


\subsection{Objetivos Específicos}

Entre os diversos objetivos específicos e objetos de estudo do projeto estão:

\begin{itemize}

    \item {Trabalhar com robótica;}

    \item {Trabalhar com sensores diversos;}

    \item {Comunicar diferentes dispositivos eletrônicos;}

    \item {Trabalhar com inteligência artificial;}

    \item {E por fim, unir esses conhecimentos para construção do robô.}


\end{itemize}
